\documentclass{beamer}
\usepackage{beamerthemeshadow}
\usepackage[ruled]{algorithm}
\usepackage{listings}
\usepackage[font=scriptsize]{subcaption}
\usepackage[font=scriptsize]{caption}
\usepackage[noend]{algpseudocode}

\begin{document}
\title[Lock-free Transaction for Linked Data Structures]{Lock-free Transaction without Aborts for Linked Data Structures}
\author[D. Zhang]{Deli Zhang}
\institute{University of Central Florida}
\date{\today}

\begin{frame}
    \titlepage
\end{frame}

%\begin{frame}
    %\frametitle{Table of Contents}
    %\tableofcontents
%\end{frame}

\section{Background}
\subsection{Motivation}
\begin{frame} \frametitle{The Problem of Method Composition}
    Execute a sequence of operations (transaction) on a linearizable concurrent data structure 
    \begin{enumerate}
        \item Atomically, i.e., if one operation fails the entire transaction fails;
        \item In isolation, i.e., concurrent execution of transactions appears to take effect in some sequential order.
    \end{enumerate}
    Existing lock-free data structures do not readily support composing method calls into transactions.
    \begin{figure}[h]
        \centering
        \includegraphics<1>[width=0.7\textwidth]{compose.pdf}
    \end{figure}
\end{frame}

\subsection{Related Work}
\begin{frame} \frametitle{Software Transactional Memory (STM)}
    \begin{block}{}
        \small
        An STM instruments threads' memory accesses, which records the locations a thread read in a \emph{read set}, and the locations it wrote in a \emph{write set}. Conflicts are detected among the \emph{read/write sets} of different threads. In the presence of conflicts, only one transaction is allowed to commit while the others are aborted and restarted.
    \end{block}
    \begin{itemize}
        \item Memory instrumentation exhibits large overhead
        \item Low-level memory conflicts do not translate to high-level semantic conflicts, which cause excessive aborts
    \end{itemize}
    \begin{figure}[h]
        \centering
        \includegraphics<1>[width=0.6\textwidth]{conflict.pdf}
    \end{figure}
\end{frame}

\begin{frame} \frametitle{Transactional Boosting}
    \begin{block}{}
        \small
        Transaction boosting uses \emph{abstract lock} to ensure that non-commutative method calls never occur concurrently. A transaction makes a sequence of calls to the objects methods after acquiring the abstract lock associated with each call. It aborts when failed to acquire a lock and recovers from failure by invoking \emph{inverses} of already executed calls.
    \end{block}
    \begin{itemize}
        \item Use of locks degrades the progress guarantee of lock-free data structures 
        \item Upon transaction failure the rollback process causes overhead
    \end{itemize}
    \begin{figure}[h]
        \centering
        \includegraphics<1>[width=0.6\textwidth]{boosting.pdf}
    \end{figure}
\end{frame}

\begin{frame} \frametitle{Lock Inference}
    \begin{block}{}
        \small
        Automatically enforces atomicity of given code fragments (multiple calls to different data structures) by inserting pessimistic \emph{abstract locks}.
    \end{block}
    \begin{itemize}
        \item Pessimistic locking avoids rollbacks 
        \item Use of locks degrades lock-freedom
        \item Require static analysis that does not scale to large code base
    \end{itemize}
    \begin{figure}[h]
        \centering
        \includegraphics<1>[width=1\textwidth]{lock-inference.png}
    \end{figure}
\end{frame}

\section{Algorithm}
\subsection{Preliminary}
\begin{frame} \frametitle{Concurrent Sets and Maps}
Abstract Data Types
    \begin{itemize}
        \item Set: Insert(k), Delete(k), Contain(k)
        \item Map: Insert(k,v), Update(k,v), Delete(k), Get(k)
    \end{itemize}
    Node-based Linked Data Structures
    \begin{itemize}
        \item Linked List, Binary Search Trees, Skiplist, MDList
        \item Common fields: $key,val,next[]$
        \item Lock-free implementations synchronize different methods on different fields, e.g., insertion on $next$, update on $val$, deletion on $val$ or $next$
    \end{itemize}
    Key Challenges
    \begin{itemize}
        \item Detect conflict at semantic level
        \item Eliminate aborts due to access conflict
        \item Provide lock-free progress
    \end{itemize}
\end{frame}

\subsection{Observation}
\begin{frame} \frametitle{Asymmetrical Update and Deletion}
    \begin{itemize}
        \item Update can be done by inserting new nodes with updated value, which replaces the target node
            \begin{exampleblock}{}
            \center
            \includegraphics<1>[width=0.7\textwidth]{update.pdf}
        \end{exampleblock}
    \item Deletion can be done by inserting new nodes (logically marked) that replaces the deleted nodes
        \begin{exampleblock}{}
            \center
            \includegraphics<1>[width=0.7\textwidth]{delete.pdf}
        \end{exampleblock}
    \end{itemize}
\end{frame}

\subsection{Details}
\begin{frame} \frametitle{Lock-free Transaction}
    \begin{itemize}
        \item All write operations are done through insertion of new nodes
        \item Each node contains a \emph{Transaction Descriptor}, which stores a sequence of operation and operands as well as the transaction statues (in-progress, succeeded, failed)
        \item The status of the transaction denotes a node's logical status depending on the operation type.
            \begin{exampleblock}{}
                \center
                \includegraphics<1>[width=0.7\textwidth]{transaction.pdf}
            \end{exampleblock}
    \end{itemize}
\end{frame}

\begin{frame} \frametitle{Lock-free Transaction Cont.}
    \begin{itemize}
        \item In-progress and failed transactions do not alter the \emph{abstract state} of the data structure
        \item Transactions only abort due to unsatisfied precondition; they do not abort on conflict
        \item Conflicting transactions help delayed operations by executing subsequent operations in the descriptor
        \item No rollback is needed for aborted transactions because the inserted nodes are logically inactive
    \end{itemize}
\end{frame}


%\section{Conclusion}
%\begin{frame} \frametitle{Comments}
    %\begin{itemize}
        %\small
        %\item Todo: fix a bug that nodes are sometimes missing after insertion
        %\item Todo: integrate Sundell's approach for further testing
        %\item Insight: high dimension for more tree like behavior - faster insertions
        %\item Insight: low dimension for more list like behavior - faster deletions
        %\item Application: Priority Queue, Dictionary, Sparse vector
    %\end{itemize}
%\end{frame}

%\begin{frame}
    %\begin{center}
        %\large
%Questions? \\
%\bigskip
%Thank you!
    %\end{center}
%\end{frame}

\end{document}
